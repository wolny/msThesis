
\chapter{Conclusions}
\label{ch:Conclusions}

\section{Summary}
The aim of my Master Thesis was to find an effective fitness deterioration
algorithm as defined in chapter \ref{ch:csfdAlgorithm}, which minimizes the
chance of multiple exploration of the same solution during the course of the \textit{Cluster Supported Fitness
Deterioration} algorithm. Taking into account the results of tests described in
chapter \ref{sec:testFun} we may come to the conclusion that the goal was
reached for highly multimodal functions.

However the assumption that distribution of individuals inside a cluster
provides information about its shape oversimplifies the real nature of the evolutionary
algorithms, where the population distribution inside basins of attraction is 
very hard to predict and depends on many factors, including recombination versus
mutation rate, selection algorithm, definition of genetic operators such as
mutation and crossover. We can expect satisfactory results when
choosing proportionate selection and genetic operators which are based on normal
distribution. \textit{Fitness Proportionate Selection} usually causes faster
convergence to local solutions and normal distribution based reproduction operators tends 
to produce populations with useful information about fitness landscape.

Based on the results from chapter \ref{sec:testFun} we recommend to use
\textit{Basic scheme} (section \ref{sec:BasScheme}) of the fitness
deterioration, cause it is a great deal cheaper than \textit{Weighted scheme}
(section \ref{sec:WeightScheme}) and yields practically the same results as
\texit{Weighted crunching functions}. The latter are more accurate only for problems with many local solutions but even for such problems it is reasonable to use \textit{Basic scheme}
because it outperforms \textit{Weighted scheme} in terms of efficiency. 

The algorithm \ref{alg:IFDA} we used to prevent degradation of
regions unexplored by the algorithm and to increase the fitness deterioration accuracy
performs well for cases when the cluster is not convex.
The most valuable outcomes of this work are:
\begin{itemize}

\item
Sequential niching with fitness deterioration is one of the 
most promising stochastic strategies
for analyzing multi-modal global optimization problems in the continuous
domains embedded in vector metric spaces.

\item
Existing instances of the strategy mentioned above
exhibits several disadvantages: 
huge memory complexity of memorizing deteriorated regions
(e.g. CGS with raster clustering); 
unsatisfactory accuracy of the fitness approximation 
that lead to the incorrect deterioration and finally may lead to removing
individuals from unchecked areas or the multiple check of non-promising areas
already browsed; 
dependency on the evolutionary technique used.

\item
The discussion presented in Sections \ref{sec:Fdt}, \ref{sec:CGS}, \ref{sec:CritRem}
clearly shows the way of necessary improvements.
The proposed deterioration strategy CSFD combines the low memory complexity
of the exponential fitness improvement with the accuracy of
clustering based techniques.

\item
The CSFD performs very well for 2D complex multi-modal functions like the ones
used for testing (see Section \ref{sec:testFun}) being also well suited to detect
the basins of attraction of the local and global extrema.
Exhaustive testing for higher dimensional problems will be the subject
of future research.

\item
Performed experiments show that we can expect satisfactory results 
when the GA utilizes
genetic operators based on the normal distribution. 
Fitness proportionate selection causes convergence to local solutions 
and the normal distribution based reproduction operators tends 
to produce populations with useful information about fitness landscape.

\item
The Hierarchical Genetic Strategy (HGS)
(see \cite{SchaeferKolodziej2003},
\cite{SchaeferAdamskaTelega2004},
\citep{WierzbaSemczukKolodziejSchaefer2003})
would be very efficient 
from the standpoint of the deterioration process.
This strategy performs an efficient concurrent
search in the optimization landscape by many small populations. Creation of
these populations is governed by dependent genetic processes with low complexity.
Moreover, HGS is likely to find many solutions in a
single run of the algorithm and that the hierarchy of populations generated
by the algorithm are rapidly convergent. 


\item
High accuracy of deterioration offered by the CSFD results from
the positive synergy of two mechanisms:
clusters obtained from the modified OPTICS (see Algorithm \ref{alg:IFDA})
and improved by leading marks
approximate well basins of attraction of local/global extrema;
the form of the weighted deteriorartion function (\ref{eqn:WS1})
and form of weights (see formulas (\ref{eqn:alpha1}), (\ref{eqn:alpha2}), 
(\ref{eqn:fractures})) maximizes the effect of deterioration over the area of
cluster extensions i.e. the area of basins of extrema already recognizes
and prevent degradation of unexplored regions.



\item
Algorithm \ref{alg:IFDA} which increases 
fitness deterioration accuracy performs well also in cases when 
the clusters are not convex e.g. for Langermann function
(see Figure \ref{fig:langermannTest}).


\item
Sequential niching obtained by CSFD preserves 
the asymptotic guarantee of success.
The probability of sampling is significantly
decreased over the cluster extensions but still greater then zero, 
so this regions are not excluded from future sampling, even in case
of inaccurate deterioration.
It seems to be the advantage over the sequential niching 
based on "tabu" techniques.

\end{itemize}


\section{Future Research}

This work focuses mainly on sequential niching which uses \textit{Fitness
Deterioration} and shows the result of Basic and Weighted variants of the algorithm applied to
simple multimodal functions. Given the \textit{Sequential
niching} algorithm described in chapter \ref{ch:csfdAlgorithm}, the main
direction of the future research should be to test this algorithm using many 
different evolutionary algorithms and evolution strategy.
The most promising algorithm to choose would be \textit{Hierarchical Genetic
Strategy} (HGS) \cite{WierzbaSemczukKolodziejSchaefer2003}.

HGS is an example of parallel
genetic algorithms and it performs very well especially in case of problems with many local extrema.
Taking into account the fact that HGS is likely to find many solutions in a
single run of the algorithm and that the hierarchy of populations generated
by the algorithm are rapidly convergent it would be very efficient 
from the standpoint of the deterioration process.

There are many aspects of the implementation of \textit{CSFD} algorithm,
described in chapter \ref{ch:Implementation}, which can be improved as well. The
framework uses very simple concurrency model which may be further developed to improve the overall performance of the algorithm.
The most costly stages of the algorithm include: fitness function computations
during EA algorithm and creation of \textit{OPTICS ordering} \cite{optics}.
The former problem might be solved by introducing some of the known
parallel genetic algorithms and the latter might be tackled by proper
domain decomposition. If we want to introduce a highly concurrent EA model
again it would be best to use HGS because of its natural concurrent character.
