
\chapter{Conclusions}
\label{Conclusions}

\section{Summary}
The aim of my Master Thesis was to find an effective fitness deterioration
algorithm as defined in chapter 4, which minimizes the chance of multiple
exploration of the same solution during the course of the \textit{Sequential
niching} algorithm. Taking into account the results of tests described in
chapter 5 we may come to the conclusion that the goal was reached. However,
tests was performed only for simple
multimodal problems in which the obtained clusters were convex and provides 
a lot of information about underlying basins of attraction. Applying the
algorithm for more demanding functions should be the next step in the further
development of the algorithm.  

The assumption that distribution of individuals inside a cluster provides
information about its shape oversimplifies the real nature of the evolutionary
algorithms, where the population distribution inside basins of attraction is 
very hard to predict and depends on many factors, including recombination versus
mutation rate, selection algorithm, definition of genetic operators such as
mutation and crossover. We can expect satisfactory results when
choosing proportionate selection and genetic operators which are based on normal
distribution. \textit{Fitness Proportionate Selection} usually causes faster
convergence to local solutions and normal distribution based reproduction operators tends 
to produce populations with useful information about fitness landscape.

Based on the results from chapter 5 we recommend to use \textit{Basic scheme}
(section 4.2) of the fitness deterioration, cause it is a great deal cheaper
than \textit{Weighted scheme} (section 4.3) and yields practically the
same results as \texit{Weighted crunching functions}. The latter are more
accurate only for problems with many local solutions but even for such problems it is reasonable to use \textit{Basic scheme}
because it outperforms \textit{Weighted scheme} in terms of efficiency. 

The algorithm (see section 4.1) we used to prevent degradation of regions
unexplored by the algorithm and to increase the fitness deterioration accuracy
should perform well for cases when the cluster is not convex, provided that
individuals inside the cluster are non-uniformly distributed,
otherwise the extraction of denser clusters would not prevent the deterioration
process from loosing information contained in regions not visited
by the \textit{sequential niching} yet.

The most valuable outcomes of this work are:
\begin{itemize}
  \item definition and implementation of a hybrid metaheuristic called \textit{Sequential
  niching} which may be used for solving multimodal optimization problems.
  \item applying clustering algorithms as a reasonable stop criterion for
  real-valued evolutionary algorithms
  \item using properties of \textit{OPTICS ordering} \cite{optics} to improve
  the deterioration process (see section 4.1) so it affects only regions
  directly occupied by population of individuals
  \item definition and implementation of two variants of the fitness
  deterioration algorithm which might be used in different scenarios depending
  on the expected speed and accuracy of the global optimization
  \item definition and implementation of the \textit{Crunching function
  adjustment} algorithm which improves the efficiency of the fitness
  deterioration
  \item creation of an extensible, lightweight framework which implements all of
  the algorithms and ideas presented in this work and which can be used for
  further research
\end{itemize}


\section{Future Research}

This work focuses mainly on the \textit{Fitness Deterioration} algorithm and
shows the result of Basic and Weighted variants of the algorithm applied to
simple multimodal functions (see chapter 4). Given the \textit{Sequential
niching} algorithm described at the beginning of this paper (chapter 2), the main
direction of the future research should be to test this algorithm using many 
different evolutionary algorithms and evolution strategy.
The most promising algorithm to choose would be \textit{Hierarchical Genetic
Strategy} (HGS) \cite{hgs}.

The Hierarchical Genetic Strategy (HGS) performs efficient concurrent
search in the optimization landscape by many small populations. The creation of
these populations is governed by dependent genetic processes with low complexity
\cite{hgs}. HGS is an example of parallel genetic algorithms and it performs
very well especially in case of problems with many local extrema.
Taking into account the fact that HGS is likely to find many solutions in a
single run of the algorithm and that the hierarchy of populations generated
by the algorithm are rapidly convergent it would be very efficient 
from the standpoint of the deterioration process.

There are many aspects of the implementation of \textit{Sequential
niching} algorithm, described in chapter 6, which can be improved as well. 
The framework uses very simple concurrency model which may be further
developed to improve the overall performance of the algorithm.
The most costly stages of the algorithm include: fitness function computations
during EA algorithm and creation of \textit{OPTICS ordering} \cite{optics}.
The former problem might be solved by introducing some of the known
parallel genetic algorithms and the latter might be tackled by proper
domain decomposition. If we want to introduce a highly concurrent EA model
again it would be best to use HGS because of its natural concurrent character.
