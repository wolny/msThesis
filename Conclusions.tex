
\chapter{Conclusions}
\label{Conclusions}

\section{Summary}
TODO: what did I do

\section{Future Research}

This work focuses mainly on the \textit{Fitness Deterioration} algorithm and
shows the result of Basic and Weighted variants of the algorithm applied to
simple multimodal functions (see chapter 4). Given the \textit{Sequential
niching} algorithm described at the beginning of this paper (chapter 2), the main
direction of the future research should be to test this algorithm using many 
different evolutionary algorithms and evolution strategy.
The most promising algorithm to choose would be \textit{Hierarchical Genetic
Strategy} (HGS) \cite{hgs}.

The Hierarchical Genetic Strategy (HGS) performs efficient concurrent
search in the optimization landscape by many small populations. The creation of
these populations is governed by dependent genetic processes with low complexity
\cite{hgs}. HGS is an example of parallel genetic algorithms and it performs
very well especially in case of problems with many local extrema.
Taking into account the fact that HGS is likely to find many solutions in a
single run of the algorithm and that the hierarchy of populations generated
by the algorithm are rapidly convergent it would be very efficient 
from the standpoint of the deterioration process.

There are many aspects of the implementation of \textit{Sequential
niching} algorithm, described in chapter 6, which can be improved as well. 
The framework uses very simple concurrency model which may be further
developed to improve the overall performance of the algorithm.
The most costly stages of the algorithm include: fitness function computations
during EA algorithm and creation of \textit{OPTICS ordering} \cite{optics}.
The former problem might be solved by introducing some of the known
parallel genetic algorithms and the latter might be tackled by proper
domain decomposition. If we want to introduce a highly concurrent EA model
again it would be best to use HGS because of its natural concurrent character.
