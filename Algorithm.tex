
\chapter{Cluster Supported Fitness Deterioration Algorithm}
\label{ch:csfdAlgorithm}

We suggest another approach to sequential niching
which exploits some of the ideas from \cite{Obuchowicz1997},
\cite{SchaeferAdamskaTelega2004}, \cite{SchaeferAdamska2004} and which
tries to overcome problems specified in Section \ref{sec:CritRem}.
This strategy is called \textit{Cluster Supported
Fitness Deterioration} (CSFD).

\section{Cluster vs. cluster extension}
\label{sec:ClystvsExt}

Before we dive into the detailed description of the CSFD algorithm
let us make a clear distinction between two important notions of the
\textit{cluster extension} and the \textit{cluster} itself, which are
necessary for further research.

\vspace{5pt}
\begin{definition}
\label{def:cluster}
We will call by \textit{cluster} $C$ a selected subset of the population $P$ 
(the mulitset of individuals) returned by GA.
We assume that each cluster will be disjoint with any other cluster from $P$. 
(i.e. when $C_i, C_k$ are clusters, then
$i \neq k \Longrightarrow C_i \cap C_k = \emptyset$). 
\end{definition}
\vspace{5pt}

The individuals are assigned to the clusters using the method
described in Chapter \ref{ch:Clustering}. 
Generally, individuals that belong to a
particular cluster $C$ have to be densely distributed and well separated from
individuals from other clusters.

\vspace{5pt}
\begin{definition}
\label{def:cluster_extension}
The \textit{extension} $CE$ of the cluster $C$ is the connected and regular 
subset of the admissible domain $\mathcal{D}$
with the positive measure, containing the cluster points, i.e., 
each element from $C$ belongs to $CE$.
Regular set means here the set with the Lipshitz boundary 
(see e.g. \cite{Zeidler1985}).
\end{definition}
\vspace{5pt}

Cluster extensions may be obtained in many different ways.
One of them is the raster procedure introduced by
\cite{Telega1999} described in Section \ref{sec:CGS}.

Here we will use the cluster extensions $CE$ being the ellypsoid
whose middlepoint is located at the cluster $C$ centroid
and the measures of diversity 
of individuals inside $C$ (e.g. in the form of standard deviation)
in order to determine the length and the direction of its axes
(see Chapter \ref{ch:fitnessDet}).

The CSFD strategy runs the GA and then distinguishes clusters from the resulting
population and constructs their extensions in each iteration.
The information contained in each cluster allows for 
effective finding the good approximation of the 
local minimizer $x^+$ contained in its extension
(e.g. by running the accurate local method starting from the best fitted
individual in the cluster).
Another advantage is the ability to approximate the shape and the volume of
basins of attraction $\mathcal{B}_{x^+}$ by cluster extensions.
This information is utilized by the fitness deterioration
technique (see Section \ref{sec:fitnessDet}). 
It might be also helpful for the stability analysis of $x^+$.



It may happen that cluster extensions obtained from two or more clusters
lie inside the same basin of attraction. This may happen when in one iteration
we manage to deteriorate only a part of the basin of attraction and in the next
iteration we obtain the cluster inside other part of the basin.  
We will use the procedure suggested by \cite{Telega1999} in order to detect
such situation starting a single local, cheep method from each new
cluster extension.
If the method converges inside one of existing cluster extensions $CE$,
then $CE$ and $CE'$ are joined
(which indicates the fact that both $CE'$ and $CE$ lie in
the same basin).

\section{CSFD algorithm description}
\label{sec:algDesc}

In principle the algorithm works like the CGS strategy  presented in 
\cite{SchaeferAdamskaTelega2004}. It uses GA to obtain
a random sample (population of individuals) from the domain $\mathcal{D}$.
If the problem space contains some robust solution located inside broad
basins of attraction, it is very likely that individuals returned from
the GA are gathered inside one or more of such basins, so the next step of the
algorithm is to apply the clustering algorithm to distinguish groups of
individuals (clusters) from the population. Under the assumption that
distribution of individuals inside a given cluster provides information
about the topology of the basin in which the cluster resides,
the algorithm approximate the basin using this information in order to 
deteriorate the fitness over the basins area
and thus to prevent convergence to the same solutions multiple times.

\subsection{CSFD pseudo-code}
\label{sec:CSFDalgorithm}


The CSFD strategy takes a hybrid approach which uses an arbitrary 
GA to obtain a random sample from which it tries 
to extract as much information as possible in order to to find 
approximations of basins of attraction. 
The only need for the GA is to be well tuned to the GOP to be solved,
i.e., the population has to concentrate in a basin of attraction
of at least one local robust minimizer (see \cite{Schaefer2007}).
Then the fitness deterioration is applied in localized 
areas to prevent exploration of the
same basins multiple times during the course of the search.
Algorithm \ref{alg:CSFD} shows the general idea behind the 
Cluster Supported Fitness Deterioration. 
The algorithm components will be described in a top-down
manner in next Sections, here we provide the general overview only.
 

\begin{algorithm}
\caption{Draft of the CSFD strategy}
\label{alg:CSFD}
\begin{algorithmic}[1]
\STATE {$CL \leftarrow \emptyset$}
\WHILE{$i < maxGenerationNumber$}
	\STATE {$execute(GA)$}
	\STATE {$pop \leftarrow getPopulation(GA)$}
	\STATE {$clusterStruct \leftarrow$ $extractClusterStruct(pop)$}
	\IF{$noClusters(clusterStruct)$}
		\RETURN
	\ENDIF
	\STATE {$detFitness \leftarrow$}\\
	$performFitnessDet(clusterStruct,currentFitness)$
	\IF{$quality(detFitness,currentFitness,pop)<th$}
		\RETURN
	\ENDIF
	\STATE {$CL \leftarrow CL \cup clusters$}
	\STATE {$updateFitness(detFitness)$}
	\STATE {$popSize = popSize + indNum$}
\ENDWHILE
\end{algorithmic}
\end{algorithm}

The condition in "while" statement (line 2, Algorithm \ref{alg:CSFD})
should be treated as a control statement rather than the real termination criterion. 
The CSFD stopping criterion is based on the conditions checked inside the main loop 
(line 6, Algorithm \ref{alg:CSFD}).
If the clustering algorithm has found no group of similar individuals 
or the deterioration has a low quality, CSFD is stopped and returns
all clusters found.

In each iteration we execute the   
GA (line 3, Algorithm \ref{alg:CSFD}) which is
treated as a black-box algorithm, i.e. we may do not need to modify or know the
implementation of the GA used. Then we take the population returned by
the GA and extract so called \textit{clustering structure} (described 
with details in Section \ref{sec:OPTICS}) which contains the information about 
clusters and their internal mean densities.
Note that so far we have not clustered the population returned by
the GA, instead we extract the information about the internal clustering structure
for further processing.
If the clustering structure contains promising clusters then we perform the
\textit{fitness deterioration} -- the process of degradation of the fitness
landscape in areas occupied by clusters of individuals which are assumed
to agglomerate inside basins of attraction. 
In the line 9, Algorithm \ref{alg:CSFD} we execute the complex
procedure \textit{performFitnessDet} (described in details in Section
\ref{sec:fitnessDetAndClustering}) which actually performs the
\textit{clustering} and \textit{fitness deterioration}. Next in the line 10, 
Algorithm \ref{alg:CSFD} we check if the new fitness 
(returned by the procedure \textit{performFitnessDet}) fulfills the 
quality requirements described in Section \ref{sec:fitnessDetAndClustering}. 
Finally, we save the clusters returned by the deterioration process and update 
the fitness function for further iterations (lines 13, 14, Algorithm \ref{alg:CSFD}).
Exploratory capabilities can be increased during later iterations by
increasing the population size of the GA (line 15, Algorithm \ref{alg:CSFD}). 
The CSFD strongly depends on the clustering algorithm used as a way to find
parts of basins of attraction and as a global termination criterion for 
the CSFD algorithm (see Section \ref{sec:TermConditions}).



\subsection{Termination conditions}
\label{sec:TermConditions}



Two types of stopping and termination criterions 
are used by CSFD: 

\begin{itemize}
  
  \item 
  \textit{Local termination criterion} which is used as a stoppng criterion
  for the GA inside the main loop of the CSFD.  
  
  \item 
  \textit{Global stopping criterion} which is based on the result of the
  clustering algorithm applied to the population returned by the GA 
  and which finishes the whole CSFD stategy.

\end{itemize}


The termination criterion in classic evolutionary algorithms is hard to
define and very often problem dependent, as we do not
have any global information about the fitness landscape and therefore we can only 
compare one solution to another previously found.
For a \textit{local termination criterion} we may use some of the 
standard well-known stopping criteria for evolutionary algorithms like: 

\begin{itemize}
  
  \item 
  \textit{Expected first hitting time (FHT)} (see e.g. \cite{PardalosRomeijn2002})
  which tries to set meaningful upper bounds for the number of iterations 
  required to reach the sufficiently small neighborhood of solution.
  
  \item 
  \textit{Efficiency measures} (see e.g. \cite{Goldberg1989}) 
  e.g. \textit{Running Mean}, the difference between the current 
  best objective value found and the average of the best objective 
	values of the last $t$ generations is equal or less than a given threshold
	$\epsilon$. 
	 
\end{itemize}

In terms of the \textit{global stopping criterion} we  
focus on the distribution of individuals
in the admissible domain $\mathcal{D}$. 
We follow the idea introduced by \cite{Telega1999}
and proved by \cite{SchaeferAdamska2004} 
to be successful for multi-modal problems with robust solutions.


The global stopping criterion for CSFD algorithm is based on the clustering analysis 
and defined as follows:

\vspace{5pt}
\noindent
\textit{The CSFD algorithm is being terminated if the clustering process applied to
the population of individuals returned by the genetic algorithm
returns no clusters or the quality measure of the fitness
deterioration performed on found clusters is too low.}
\vspace{5pt}

The clustering, which is performed in every iteration of the CSFD algorithm,
gives us some clues about the global characteristics of the fitness landscape 
i.e. when the clustering  algorithm performed on the final population finds
nothing it is very likely that in previous iterations we have deteriorated the fitness landscape in places where the most desirable solutions reside and there is no use in
continuing the searching process.
This is considered to be true because we are looking for robust solutions which are resistant to noise and lie in basins of attraction which are significantly wide and deep.
The population of GA is likely to converge to such solutions, so
having found no clusters of individuals after performing 
the sufficient number of GA epochs  
shows that the population would not converge to any robust solution.

Such kind of condition may be precisely formulated in terms of the convergence
of sampling measures and partially verified in case if the genetic engine
is SGA with the focusing heuristic (see \cite{Telega1999},
\cite{Schaefer2007}, \cite{SchaeferJablonski2001}).
