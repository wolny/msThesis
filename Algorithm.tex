
\chapter{Algorithm}
\label{Algorithm}

\section{A Hybrid Approach}

Our deterioration algorithm may be easily incorporated in existing optimization
methods. This can be expressed as a general hybrid approach to global
optimization in which we do not change the implementation of the used EA
but treat it as an integral element of our iterative process.
This can be visualized by the following pseudo code:




\begin{algorithmic}[1]
\WHILE{$i < getIterationCount()$}
	\STATE {$execute(evolutionaryAlgorithm)$}
	\STATE {$population=getPopulation(evolutionaryAlgorithm)$}
	\STATE {$clusters=cluster(population)$}
	\IF{$clusters.isEmpty()$}
		\STATE {$break$}
	\ENDIF
	\STATE {$detFitness=performCrunching(clusters,currentFitness)$}
	\STATE {$saveClusters(clusters)$}
	\STATE {$updateFitness(detFitness)$}
\ENDWHILE
\STATE {$execute(evolutionaryAlgorithm)$}
\STATE {$extractBestClusters()$}
\end{algorithmic}




The condition in while statement should be treated as control statement rather
than the real termination criterion. It may be useful in cases where the
clustering algorithm is misconfigured and always returns some clusters in which
case the condition would prevent our algorithm to run endlessly.
Our actual stop criterion is based on the condition inside the loop (line 5): if the clustering algorithm did not find any group of similar individuals or the group has low quality measures it jumps out
of the loop and then we perform the last invocation of the EA in order to
increase the probability of finding the local optimum which has not been explored during iterations.

Using this general scheme has one important advantage: we do not have to change
existing implementation of a given EA in contrast to standard niching
methods which must be incorporated directly in the evolutionary algorithm. But
what is more important, it provides a reasonable stop criterion which in this case is based 
on the quality of clusters returned by the clustering algorithm.
 
In the subsequent chapters we will describe components  of the general
algorithm in more details.
In chapter 3 we will discuss the clustering method used in our algorithm and 
how does it influence the deterioration process. Chapter 4 describes
deterioration algorithm in more detail and chapter 5 shows the results 
of the tests.

